% coding: utf-8
\documentclass [a4paper,abstracton,titlepage]{scrartcl}
\pagestyle{plain}
\usepackage{makeidx}
\usepackage[table]{xcolor}
\usepackage{color}
\definecolor{LinkColor}{rgb}{0.33,0.42,0.18}
\definecolor{TableEvenColor}{rgb}{0.93,1,0.8}
\definecolor{TableOddColor}{rgb}{0.93,1,1}
\definecolor{TableBorderColor}{rgb}{0.55,0.67,0.73}
\definecolor{ListingBorderColor}{rgb}{0.55,0.55,0.55}
\definecolor{ListingBackgroundColor}{rgb}{0.95,0.95,0.95}
\definecolor{SidebarBorderColor}{rgb}{0.95,0.95,0.95}
\definecolor{SidebarBackgroundColor}{rgb}{1,1,0.93}
\usepackage[spanish]{babel}
\usepackage[
    pdftex,
    pdftitle={Informe Final - Compiladores e Interpretes},
    pdfauthor={Touceda, Tomás LU 84024},
    backref,
    pagebackref,
    breaklinks=true,
    unicode
    ]
    {hyperref}
\usepackage{enumerate}
\usepackage{graphicx}
\usepackage{longtable}
\usepackage[T1]{fontenc}
\usepackage{ucs}
\usepackage[utf8x]{inputenc}
\usepackage{textcomp}
\usepackage{alltt}
\usepackage{listings}
\usepackage{moreverb}
\usepackage{upquote}

\lstset{basicstyle=\footnotesize\ttfamily,showstringspaces=false,breaklines,frame=single, rulecolor=\color{ListingBorderColor}, xleftmargin=0cm, linewidth=0.95\textwidth}

\setlength{\parskip}{1ex plus 0.5ex minus 0.2ex}

\title{Informe final - Compiladores e Intérpretes\\Manual del usuario}
\author{Touceda, Tomás LU 84024}
\publishers{\begin{tabular}{ll}    \end{tabular}}

\makeindex

\begin{document}

\maketitle

\tableofcontents

\newpage

\hypertarget{_especificación_e_instrucciones_de_uso}{}
\section{Instrucciones de uso}

\subsection{Compilador}

El compilador se debe ejecutar de la siguiente manera:

\begin{alltt}C:\textbackslash{}...\textbackslash{}\textgreater{} python src/compilor.py \textless{}archivo\_{}entrada\textgreater{} [\textless{}archivo\_{}salida\textgreater{}]\end{alltt}

La especificación del archivo de salida es opcional.
Si se omite, el compilador generará el archivo de código intermedio con el mismo nombre del archivo de entrada, pero con una extensión ``.asm'' en vez de ``.java''.
Si no se especifica un archivo de entrada, o se pasan más argumentos que el
archivo de entrada y el de salida, el compilador mostrará un mensaje
explicando el uso por linea de comando.

\subsection{CeIVM}

Para ejecutar el código intermedio generado por el compilador en la máquina virtual de MiniJava, se deberá utilizar el siguiente comando:

\begin{alltt}C:\textbackslash{}\textgreater{} java -jar CeIVM-cei2011-v1.0.1/CeIVM-cei2011.jar \textless{}archivo\_{}entrada\textgreater{}\end{alltt}

\section{Especificación de MiniJava}

\subsection{Tipos primitivos}

MiniJava cuenta con los siguientes tipos predefinidos: int, char y boolean.

\subsection{Tipos built-in}

Además de los tipos primitivos, se proveen tres tipos ``built-in'':

\begin{itemize}
 \item Object: Toda clase definida hereda de la clase Object. No posee ningún método, existe simplemente para dar flexibilidad al programador a la hora de realizar programación genérica.
 \item String: Se cuenta con la clase String, cuyos objetos solo pueden ser creados estableciendo el texto del string entre comillas dobles. Más información a continuación.
 \item System: Se cuenta con la clase System, de la cual no se pueden crear objetos sino que se pueden utilizar alguno de sus métodos estáticos. Más información a continuación.
\end{itemize}

\minisec{String}

La clase String cuenta con los siguientes métodos de instancia:

\begin{itemize}
 \item \texttt{char charAt(int index)}: Devuelve el caracter que ocupa la posición en el String o indicada por index.
 \item \texttt{String concat(String str)}: Concatena str al final del String y lo devuelve.
 \item \texttt{boolean equals(String str)}: Devuelve verdadero si str y el String contienen la misma secuencia de caracteres, y falso en caso contrario.
 \item \texttt{int length()}: Devuelve la longitud del String.
\end{itemize}

\minisec{System}

La clase System cuenta con los siguientes métodos estáticos:

\begin{itemize}
 \item \texttt{static int read()}: lee el próximo byte del stream de entrada estándar
 \item \texttt{static void print(boolean b)}: imprime un boolean por salida estándar
 \item \texttt{static void print(char c)}: imprime un char por salida estándar
 \item \texttt{static void print(int i)}: imprime un int por salida estándar
 \item \texttt{static void print(String s)}: imprime un String por salida estándar
 \item \texttt{static void println()}: imprime un separador de línea por salida estándar finalizanado la línea actual 
 \item \texttt{static void println(boolean b)}: imprime un boolean por salida estándar finalizanado la línea actual
 \item \texttt{static void println(char c)}: imprime un char por salida estándar finalizanado la línea actual
 \item \texttt{static void println(int i)}: imprime un int por salida estándar finalizanado la línea actual
 \item \texttt{static void println(String str)}: imprime un String por salida estándar finalizanado la línea actual
\end{itemize}

\subsection{Tipos definidos por el usuario}

MiniJava provee la posibilidad de declarar clases nuevas que posean o no herencia explícita, en las cuales se pueden declarar constructores, métodos estáticos o de instancia, y atributos estáticos o de instancia. La visibilidad de las declaraciones dentro de una clase podrá ser $public$ o $protected$ únicamente, todo el resto de los modificadores son inválidos.

Dentro de los métodos y constructores, se podrán definir variables locales. No se podrá usar ningún tipo de manejo de excepciones.

Las clases pueden ser únicamente $public$. No es posible utilizar declaraciones de $package$ o el uso de $import$. Las clases tampoco pueden ser abstractas ni interfaces.

Únicamente uno de los tipos definidos por el usuario debe contener un método static cuyo signature sea \texttt{public static void main()}.

\subsection{Sentencias y expresiones}

Las posibles sentencias y expresiones que se pueden utilizar en MiniJava son las siguientes:

\begin{itemize}
 \item Asignaciones.
 \item Invocación a métodos.
 \item Creación de instancia de clase.
 \item Sentencias compuestas.
 \item Sentencias condicionales como \texttt{if (expr) then sent} o \texttt{if (expr) sent else sent}.
 \item Sentencias de repetición \texttt{while}.
 \item Sentencias de retorno de métodos.
 \item Expresiones aritméticas con los operadores: +, -, /, * y \%.
 \item Operaciones booleanas con los operadores: \&\&, ||, !, <, >, <=, >=, ==, !=.
 \item Concatenación de Strings con el operador +.
\end{itemize}

\end{document}